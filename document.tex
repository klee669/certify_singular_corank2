\documentclass[12pt,oneside,reqno]{amsart}


\pagenumbering{arabic}
\usepackage{amsmath}
\usepackage{amssymb}
\usepackage{latexsym}
\usepackage{amsfonts,setspace}
\usepackage{fullpage}
\usepackage{indentfirst}
\usepackage{algorithm,algpseudocode}
\usepackage{color}
\usepackage{changepage}
\usepackage{paralist}
%\usepackage{enumitem}
\usepackage{tikz}
\usepackage{bm}


\usepackage{kpfonts}
\usepackage[T1]{fontenc}


% Remove the following before submission
\usepackage{todonotes}
\usepackage{refcheck}

\usepackage{makecell}
\usepackage{pict2e}
\usepackage{amsthm}
\theoremstyle{definition}
\newtheorem{thm}{Theorem}
\newtheorem{rmk}[thm]{Remark}
\newtheorem{prop}[thm]{Proposition}
\newtheorem{cor}[thm]{Corollary}
\newtheorem{lem}[thm]{Lemma}
\newtheorem{lemdef}[thm]{Lemma/Definition}
\newtheorem{exm}[thm]{Example}
\newtheorem{defi}[thm]{Definition}
\renewcommand{\qedsymbol}{$\#$}

% \setlength{\textwidth}{6.5in}
% \setlength{\textheight}{9in}
% \setlength{\topmargin}{-0.5in}
% \setlength{\evensidemargin}{0in}
% \setlength{\oddsidemargin}{0in}
% \setlength{\headheight}{0in}
% \setlength{\headsep}{0in}


\setlength{\abovedisplayskip}{3mm}
\setlength{\belowdisplayskip}{3mm}
\setlength{\abovedisplayshortskip}{0mm}
\setlength{\belowdisplayshortskip}{2mm}
\setlength{\baselineskip}{12pt}
\setlength{\normalbaselineskip}{12pt}

\newcommand{\anton}[1]{{\color{red} (Anton: #1)}}
\newcommand{\kisun}[2]{{\color{blue} (Kisun: #2)}}

\newcommand{\blank}{\underline{~~~~~~~~~}}
\newcommand{\RR}{\mathbb{R}}
\newcommand{\CC}{\mathbb{C}}
\newcommand{\QQ}{\mathbb{Q}}
\newcommand{\ZZ}{\mathbb{Z}}
\newcommand{\PP}{\mathbb{P}}
\newcommand{\FF}{\mathbb{F}}
\newcommand{\KK}{\mathbb{K}}

\newcommand{\calB}{\mathcal{B}}
\newcommand{\calC}{\mathcal{C}}

\newcommand{\appx}{{\widetilde x}}
%\newcommand{\pointnorm}[1]{\|#1\|_1}
\newcommand{\pointnorm}[1]{\|(1,#1)\|}

\newcommand{\bu}{\mathbf{u}}
\newcommand{\bv}{\mathbf{v}}
\newcommand{\bx}{\mathbf{x}}
\newcommand{\by}{\mathbf{y}}
\newcommand{\bb}{\mathbf{b}}
\newcommand{\bc}{\mathbf{c}}
\newcommand{\bp}{\mathbf{p}}
\newcommand{\bq}{\mathbf{q}}
\newcommand{\be}{\mathbf{e}}
\newcommand{\bzero}{\mathbf{0}}
\DeclareMathOperator{\Nul}{Nul}
\DeclareMathOperator{\Mod}{mod}
\DeclareMathOperator{\ord}{ord}
\DeclareMathOperator{\GL}{GL}
\DeclareMathOperator{\SL}{SL}
\DeclareMathOperator{\Hom}{Hom}
\DeclareMathOperator{\End}{End}
\DeclareMathOperator{\Ind}{Ind}
\DeclareMathOperator{\im}{im}
\DeclareMathOperator{\tr}{tr}
\DeclareMathOperator{\erf}{erf}
\DeclareMathOperator{\diffop}{\mathbf{d}^\mathbf{\alpha}_x}

\title{Computing simple triple zeros of polynomial system : corank 2 case}

\begin{document}


\maketitle
\subsection{The main goal}
Dedieu and Shub \cite{dedieu2001simple} suggests a numerical criterion for multiple zeros of a system of analytic functions for multiplicity $2$ case. We want to generalize these results into the simple triple zeros of corank $2$ case for polynomial system.

More precisely, for a square polynomial system $f:\mathbb{C}^n\rightarrow \mathbb{C}^n$ with small $\|f(x)\|,\|Df(x)\|$ and some additional conditions, we want to conclude that $x$ is close to (or a part of) cluster of $3$ points in a disk of some radius $r$ and that we can correspond $(f,x)$ to $(\tilde{f},\tilde{x})$ where $\tilde{f}$ is a polynomial system with a solution $\tilde{x}$ with multiplicity $3$ and $\dim\ker D\tilde{f}(\tilde{x})=2$.

\subsection{Local dual space}
Let $\mathbf{d}^\mathbf{\alpha}_x :\mathbb{C}[X]\rightarrow \mathbb{C}$ denote the differential functional defined by
\[\diffop(g)=\frac{1}{\alpha_1!\cdots \alpha_n!}\frac{\partial^{|\mathbf{\alpha}|}g}{\partial {X_1}^{\alpha_1}\cdots \partial{X_n}^{\alpha_n}}(x), \quad g\in \mathbb{C}[X]\]
where $x\in \mathbb{C}^n$ and $\mathbf{\alpha}\in \mathbb{N}^n$.

%Let $I_f$ be an ideal generated by a polynomial system $f=\{f_1,\dots,f_n\}$. If we let $\mathfrak{D_x}=span_\mathbb{C}\{\diffop\}$, then the local dual space $\mathcal{D}_{f,x}$ of $I_f$ at a given isolated singular solution 
\begin{itemize}
	\item $I_f$ : an ideal generated by a polynomial system $f=\{f_1,\dots, f_n\}:\mathbb{C}^n\rightarrow\mathbb{C}^n$.
	\item $x\in \mathbb{C}^n$ : an isolated multiple root of $f$.
	\item $Q$ : the primary component of $I_f$ at $x$.
	\item $\mathfrak{D}_x=span_\mathbb{C}\{\diffop\}$.
	\item $\mathcal{D}_{f,x}=\{\Lambda \in \mathfrak{D}_x \mid \Lambda(g)=0, g\in I_f\}$ : the local dual space of $I_f$ at a given isolated singular solution $x$.
	\item $\mathcal{D}_{f,x}^{(k)}$ : the subspace of $\mathcal{D}_{f,x}$ with functionals of orders bounded by $k$.
\end{itemize}
Using these notations, we define
\begin{enumerate}
	\item breadth $\kappa = \dim\left(\mathcal{D}_{f,x}^{(1)}\setminus\mathcal{D}_{f,x}^{(0)}\right)$,
	\item depth $\rho = \min\left(\left\{k\mid \dim\left(\mathcal{D}_{f,x}^{(k+1)}\setminus\mathcal{D}_{f,x}^{(k)}\right)=0\right\}\right)$,
	\item multiplicity $\mu = \dim \left(\mathcal{D}_{f,x}^{(\rho)}\right)$.
\end{enumerate}
\begin{lemdef}\cite[Lemma 4.1]{hauenstein2015certifying}
	Let $f,x,Q,\mathcal{D}_{f,x},\mu, \rho$ be as above. Then, there is a primal-dual basis pair of the local ring $\mathbb{C}[X]/Q$ with the following properties:
	\begin{itemize}
		\item The {\it primal basis} of the local ring $\mathbb{C}[X]/Q$ has the form
		\[B:=\{(X-x)^{\alpha_1},\dots, (X-x)^{\alpha_\mu}\}.\]
		We can assume that $\alpha_1=0$ and that the monomials in $B$ are {\it connected to $1$} (see \cite{mourrain1999new}). Let $E:=\{\alpha_1,\dots, \alpha_\mu\}$ be the set of exponents in $B$.
		\item There is a unique {\it dual basis} in $\mathcal{D}_{f,x}$ orthogonal to $B$, i.e., the basis elements are given in the following form:
		\begin{eqnarray*}
		\Lambda_0 & = & 1\\
		\Lambda_1 & = & \frac{1}{\alpha_2!}\mathbf{d}_x^{\alpha_2}+\sum\limits_{\substack{|\beta|\leq \rho\\ \beta \notin E}}a_{\alpha_2,\beta}\mathbf{d}^\beta_x\\
		&\vdots &\\
		\Lambda_{\mu-1} & = & \frac{1}{\alpha_{\mu}!}\mathbf{d}_x^{\alpha_\mu}+\sum\limits_{\substack{|\beta|\leq \rho\\ \beta \notin E}}a_{\alpha_\mu,\beta}\mathbf{d}^\beta_x		
		\end{eqnarray*}
		\item We have $0=\ord(\Lambda_0)\leq \cdots \leq \ord(\Lambda_{\mu-1})$, and for all $0\leq k \leq \rho$, we have
		\[\mathcal{D}_{f,x}^{(k)}=span_\CC\{\Lambda_j\mid \ord(\Lambda_j)\leq k\}\]
	\end{itemize}
\end{lemdef}


We deal with simple triple zeros satisfying $f(x)=0, \dim \ker Df(x)=2$. Therefore, $\kappa=2, \mu=3, \rho =1$.
%\[\dim\left(\mathcal{D}_{f,x}^{(1)}\setminus\mathcal{D}_{f,x}^{(0)}\right)=2.\]
It means that we have 
\[\mathcal{D}_{f,x}=span_\mathbb{C}\{\Lambda_0,\Lambda_1,\Lambda_2\}\]
where $\Lambda_0=1$ and $\deg(\Lambda_i)=1$ for $i=1,2$. Also, it is always possible to perform unitary transformations to obtain an equivalent polynomial system whose Jacobian matrix at the simple multiple zero has a normalized form.
\begin{defi}
	For a polynomial function $f=\{f_1,\dots, f_n\}:\mathbb{C}^n\rightarrow\mathbb{C}^n$, $Df(x)$ has a normalized form if 
	\[Df(x)=\begin{bmatrix}
	D\hat{f}(x) & 0 & 0\\
	0 & 0 & 0 \\
	0 & 0 & 0
	\end{bmatrix},\label{eq:normalizedForm}\]
	where $D\hat{f}(x)$ is the nonsingular Jacobian matrix of polynomials $\hat{f}=\{f_1,\dots, f_{n-2}\}$ with respect to $\hat{X}=\{X_1,\dots, X_{n-2}\}$.
\end{defi}


If we assume this normalized form, then we are able to have elements of dual basis $\Lambda_1=d_{n-1}$ and $\Lambda_2=d_n$. Then, we can define the simple triple zero with corank $2$ in the following way:
\begin{defi}
 Let $f:\mathbb{C}^n\rightarrow\mathbb{C}^n$ be a polynomial system which has a normalized form and suppose that $f(x)=0$. Then, $x$ is a {\it simple triple zero with corank $2$} for $f$ if
 \begin{itemize}
 	\item[(A)] $\dim\ker Df(x)=2$
 	\item[(B)] $d_k(f_j)=0$ for $k=n-1,n$ and $j=n-1,n$,
 	\item[(C)] $d_{n-1}^2(f_{n-1})\ne0$ and $d_{n}^2(f_{n})\ne0$.
 	\end{itemize}
	
\end{defi}


\subsection{Results}
Let $x$ be a simple triple zero of $f$ and $Df(x)$ has the normalized form \ref{eq:normalizedForm}. Then, we have 
\[\frac{\partial f_i(x)}{\partial X_j}=0,\quad \frac{\partial f_j(x)}{\partial X_i}=0, \quad 1\leq i\leq n,\quad j=n-1,n\]
and
\[d_k(f_j)=0,\quad d_k^2(f_j)\ne 0\quad \text{for }j,k=n-1,n.\]

Then, we redefine the parameter $\gamma$ in $\alpha$-theory in the following way:
$$\gamma_2(f,x)=\max(\hat{\gamma}_2,\gamma_{2,n-1},\gamma_{2,n})$$
where
\[\hat{\gamma}_2(f,x)=\max\left(1,\sup\limits_{k\geq 2}\left\|D\hat{f}(x)^{-1}\frac{D^k\hat{f}(x)}{k!}\right\|^{\frac{1}{k-1}}\right),\]
and
\[\gamma_{2,j}=\max\left(1,\sup\limits_{k\geq 2}\left\|\frac{1}{d_j^2(f_j)}\frac{D^k f_j(x)}{k!}\right\|^{\frac{1}{k-1}}\right)\quad \text{ for } j=n-1,n\]


Also, we have a universal constant $d\approx0.158921$, and an invertible operator 
\[\mathcal{A}=\begin{bmatrix}
\sqrt{3}D\hat{f}(x) & 0 & 0\\
0 & \frac{1}{\sqrt{3}}d_{n-1}^2(f_{n-1}) & 0 \\
0 & 0 & \frac{1}{\sqrt{3}}d_{n}^2(f_{n})
\end{bmatrix}.\]

From now on, $\|\cdot \|$ is the usual Euclidean norm.
\begin{thm}\label{thm:tripleRootSeparationBound}
	Let $x$ be an isolated simple triple zero with corank $2$ of the polynomial system $f:\mathbb{C}^n\rightarrow\mathbb{C}^n$. Then, if we let $y$ be another zero of $f$, then
	\[\|y-x\|\geq \frac{d}{2\gamma_2^2}\]
\end{thm}

\begin{thm}\label{thm:lowerboundfy}
	Let $x$ be an isolated simple triple zero with corank $2$ of the polynomial system $f:\mathbb{C}^n\rightarrow \mathbb{C}^n$ with another solution $y$. Then, if $\|y-x\|\leq \frac{d}{4\gamma_2^2}$, then
	\[\left\|f(y)\right\|\geq \frac{d\|y-x\|^2}{\|\mathcal{A}^{-1}\|}\]
\end{thm}


Now for $R>0$, define
\begin{equation*}
d_R(f,g)=\max\limits_{\|y-x\|\leq R}\|f(y)-g(y)\|.
\end{equation*}
\begin{thm}\label{thm:multiplicityOfDiffSystem}
	Let $x$ be an isolated simple triple zero with corank $2$ of the polynomial system $f:\mathbb{C}^n\rightarrow\mathbb{C}^n$, and $0<R\leq \frac{d}{4\gamma_2^2}$. If $d_R(f,g)<\frac{dR^2}{\|\mathcal{A}^{-1}\|}$, then the sum of the mulitplicities of the zeros of $g$ in $B(x,R)$ is $3$.
\end{thm}

Now, we will state the main theorem. Let $f:\mathbb{C}^n\rightarrow\mathbb{C}^n$ be any polynomial system and $x$ be any point in $\CC^n$ such that $D\hat{f}(x)$ is invertible and $d_j^2(f_i)\ne 0$ for all $i,j=n-1,n$. We define a $n\times n$-matrix 
\[H=\begin{bmatrix}
0 & \frac{\partial\hat{f}(x)}{\partial X_{n-1}} & \frac{\partial\hat{f}(x)}{\partial X_{n}}\\
\frac{\partial f_{n-1}(x)}{\partial\hat{X}} & \frac{\partial f_{n-1}(x)}{\partial X_{n-1}} & \frac{\partial f_{n-1}(x)}{\partial X_{n}}\\
\frac{\partial f_{n}(x)}{\partial\hat{X}} & \frac{\partial f_{n}(x)}{\partial X_{n-1}} & \frac{\partial f_{n}(x)}{\partial X_{n}}
\end{bmatrix}\]
and a polynomial
\[g(X)= f(X)-f(x)-H(X-x).\]
\begin{thm}\label{thm:clusterthm}
	If we let $\gamma_2=\gamma_2(g,x)$ and \[\|f(x)\|+\|H\|\frac{d}{4\gamma_2^2}<\frac{d^3}{16\gamma_2^4\|\mathcal{A}^{-1}\|},\]
	then $f$ has three zeros (counting multiplicities) in the ball of radius $\frac{d}{4\gamma_2^2}$ around $x$.
\end{thm}
\subsection{Proofs}

For two nonzero vectors $a,b\in \mathbb{C}^n$, we define their angle by
\[d_P(a,b)=\arccos\frac{|\left<a,b\right>|}{\|a\|\|b\|}.\]
Let $y$ be another zero of $f$ and define
\[w=y-x=\left(\begin{array}{c}
\eta_1\\
\vdots\\
\eta_{n-2}\\
\zeta_{n-1}\\
\zeta_n
\end{array}\right),\quad \eta=\left(\begin{array}{c}
\eta_1\\
\vdots\\
\eta_{n-2}
\end{array}\right),\quad\eta_{n-1}=\left(\begin{array}{c}
\eta\\
\zeta_{n-1}\\
0
\end{array}\right),\quad \eta_{n}=\left(\begin{array}{c}
\eta\\
0\\
\zeta_{n}
\end{array}\right).\]
Then, for $v_{n-1}=(0,\cdots,0,1,0)^\top, v_n=(0,\dots,0,1)^\top$ and $v=\zeta_{n-1}v_{n-1}+\zeta_n v_n$, we define $\varphi=d_P(v,w)$ and $\varphi_j=d_P(v_j,w)$ for $j=n-1,n$. Then, we have
\[\|\eta\|=\|w\|\sin\varphi,\quad |\zeta_j|=\|w\|\cos\varphi_j,\quad\text{ and }\quad\|\eta_j\|=\|w\|\sin\varphi_j \quad\text{ for }j=n-1,n.\]
Also, we have $\cos^2\varphi = \cos ^2\varphi_{n-1}+\cos^2\varphi_n$.
\begin{lem}\label{lem:sinIneq}
	For $\varphi_{n-1},\varphi_n,\varphi \in [0,\frac{\pi}{2}]$,
	\[\sin\varphi_{n-1}+\sin\varphi_n\geq \sin \varphi\]
\end{lem}
\begin{proof}
	Take a square on both sides. Then,
	\begin{eqnarray*}
		\left(\sin\varphi_{n-1}+\sin\varphi_n\right)^2& = & \left(2-\cos^2\varphi_{n-1}-\cos^2\varphi_n\right)+2\sqrt{\left(1-\cos^2\varphi_{n-1}\right)\left(1-\cos^2\varphi_n\right)}\\
		&\geq & \left(2-\cos^2\varphi_{n-1}-\cos^2\varphi_n\right)\\
		& \geq & 2-\cos^2 \varphi\\
		& \geq & 1-\cos^2 \varphi\\
		& = & \sin^2 \varphi
	\end{eqnarray*}
	Therefore, the claim is proved.
\end{proof}
\begin{lem}\cite[Lemma 1]{hao2017computing}\label{lem:bigAngleCase}
	If $\gamma_2(f,x)\|w\|\leq \frac{1}{2}$, then
	\[\left\|D\hat{f}(x)^{-1}\hat{f}(y)\right\|\geq \|w\|\sin\varphi -2\hat{\gamma}_2(f,x)\|w\|^2.\]
\end{lem}
\begin{proof}
\todo[inline]{Just same as the proof in \cite{hao2017computing}.}	
\end{proof}

\begin{lem}\label{lem:inverseOperatorLowerbound}
	If $\gamma_2(f,x)\|w\|\leq \frac{1}{2}$, then 
	\[\left\|\mathcal{A}^{-1}f(y)\right\|\geq \sum\limits_{j=n-1}^n\left(\left(\frac{1}{2}\cos^2\varphi_{j}-\gamma_{2}\cos\varphi_j\sin\varphi_j-\frac{1}{2}\gamma_{2}\sin^2\varphi_j\right)\|w\|^2 -2\gamma_{2}^2\|w\|^3\right)\]
\end{lem}
\begin{proof}
	Consider the taylor expansion of $f_{n-1}(x)$. Then, we have
	{\footnotesize	\[f_{n-1}(y)=\frac{1}{2}\frac{\partial^2 f_{n-1}(x)}{\partial X_{n-1}^2}\zeta_{n-1}^2+\frac{\partial^2 f_{n-1}(x)}{\partial X_{n-1}\partial \hat{X}_n}\zeta_{n-1}\eta_n+\frac{1}{2}\frac{\partial^2 f_{n-1}(x)}{\partial\hat{X}_n^2}\eta_n^2 +\sum\limits_{k\geq 3}Df_{n-1}(x)^{-1}\frac{D^k\hat{f}_{n-1}(x)}{k!}w^k\]}
	where $\hat{X}_n$ denotes variables $X_1,\dots, X_{n-2}$ and $X_n$.
	Dividing by $d_{n-1}^2(f_{n-1})$ on both sides,
	{\footnotesize \[\frac{1}{d_{n-1}^2(f_{n-1})}f_{n-1}(y)=\frac{1}{2}\zeta_{n-1}^2+\frac{1}{d_{n-1}^2(f_{n-1})}\left(\frac{\partial^2 f_{n-1}(x)}{\partial X_{n-1}\partial \hat{X}_n}\zeta_{n-1}\eta_n+\frac{1}{2}\frac{\partial^2 f_{n-1}(x)}{\partial\hat{X}_n^2}\eta_n^2 +\sum\limits_{k\geq 3}Df_{n-1}(x)^{-1}\frac{D^k\hat{f}_{n-1}(x)}{k!}w^k\right).\]}
	Now, note that if we have $i+j=k$, then
	{\footnotesize\[\left\|\frac{\partial^k f_{n-1}(x)}{\partial X_{n-1}^i\partial \hat{X}_n^j}\right\|\leq \left\|D^k f_{n-1}(x)\right\|\]}
	Then, we have
	{\footnotesize
		\begin{eqnarray*}
			\frac{1}{2}\left|\zeta_{n-1}\right|^2 &\leq & \left|\frac{1}{d_{n-1}^2(f_{n-1})}f_{n-1}(y)\right| + \gamma_2|\zeta_{n-1}|\|\eta_n\|+\frac{1}{2}\gamma_2\|\eta_n\|^2 +\sum\limits_{k\geq 3}\gamma_2^{k-1}\|w\|^k\\
			&\leq & \left|\frac{1}{d_{n-1}^2(f_{n-1})}f_{n-1}(y)\right| + \gamma_2|\zeta_{n-1}|\|\eta_n\|+\frac{1}{2}\gamma_2\|\eta_n\|^2 +2\gamma_2^{2}\|w\|^3
		\end{eqnarray*}
	}
	The last inequality comes from the assumption that $\gamma_2(f,x)\|w\|\leq \frac{1}{2}$.
	
	Then, since $|\zeta_{n-1}|=\|w\|\cos\varphi_{n-1}, \|\eta_n\|=\|w\|\sin\varphi_{n-1}$, we have
	{\footnotesize
		\begin{eqnarray*}
			\frac{1}{2}\|w\|^2\cos^2\varphi_{n-1}& \leq & \left|\frac{1}{d_{n-1}^2(f_{n-1})}f_{n-1}(y)\right| + \gamma_2\|w\|^2\cos\varphi_{n-1}\sin\varphi_{n-1}+\frac{1}{2}\gamma_2\|w\|^2\sin^2\varphi_{n-1}+2\gamma^2\|w\|^3.
		\end{eqnarray*}
	}
	It implies that
	{\footnotesize
		\begin{eqnarray*}
			\left(\frac{1}{2}\cos^2\varphi_{n-1}-\gamma_2\cos\varphi_{n-1}\sin\varphi_{n-1}-\frac{1}{2}\gamma_2\sin^2\varphi_{n-1}\right)\|w\|^2-2\gamma_2^2\|w\|^3&\leq & \left|\frac{1}{d_{n-1}^2(f_{n-1})}f_{n-1}(y)\right| .
		\end{eqnarray*}}
		Likewise, we have
		{\footnotesize
			\begin{eqnarray*}
				\left(\frac{1}{2}\cos^2\varphi_{n}-\gamma_2\cos\varphi_{n}\sin\varphi_{n}-\frac{1}{2}\gamma_2\sin^2\varphi_{n}\right)\|w\|^2-2\gamma_2^2\|w\|^3&\leq & \left|\frac{1}{d_{n}^2(f_{n})}f_{n}(y)\right|.
			\end{eqnarray*}}
			Thus,
			$${\footnotesize
				\begin{array}{l}
				\sum\limits_{j=n-1}^n\left(\left(\frac{1}{2}\cos^2\varphi_{j}-\gamma_{2,j}\cos\varphi_j\sin\varphi_j-\frac{1}{2}\gamma_{2,j}\sin^2\varphi_j\right)\|w\|^2 -2\gamma_{2,j}^2\|w\|^3\right)\\
				\leq \left|\frac{1}{d_{n-1}^2(f_{n-1})}f_{n-1}(y)\right| + \left|\frac{1}{d_{n}^2(f_{n})}f_{n}(y)\right| \\
				\leq \sqrt{3}\left\|\left(\begin{array}{c}
				\frac{1}{3}D\hat{f}(x)^{-1}\hat{f}(y)\\
				\frac{1}{d_{n-1}^2(f_{n-1})}f_{n-1}(y)\\
				\frac{1}{d_{n}^2(f_{n})}f_{n}(y)
				\end{array}\right)\right\|\\
				\leq \left\|\left(\begin{array}{ccc}
				\frac{1}{\sqrt{3}}D\hat{f}(x)^{-1} & 0 & 0\\
				0 & \frac{\sqrt{3}}{d_{n-1}^2(f_{n-1})} & 0 \\
				0 & 0 & \frac{\sqrt{3}}{d_{n}^2(f_{n})}
				\end{array}\right)\left(\begin{array}{c}
				\hat{f}(y)\\
				f_{n-1}(y)\\
				f_n(y)
				\end{array}\right)\right\|\\
				\leq \left\|\mathcal{A}^{-1}\left(\begin{array}{c}
				\hat{f}(y)\\
				f_{n-1}(y)\\
				f_n(y)
				\end{array}\right)\right\|=\left\|\mathcal{A}^{-1}f(y)\right\|.
				\end{array}}$$
		\end{proof}
		\begin{lem}\label{lem:operatorlowerbound}
			Let $d\approx 0.158921$ be the positive root of the equation 
			\begin{equation*}
			(1-d^2)-2d\sqrt{1-d^2}-d^2 -4d=0.
			\end{equation*}
			Let $\theta$ be defined by 
			\begin{equation*}
			\sin \theta=\frac{d}{\gamma_2}.
			\end{equation*}
			Then, if $\gamma_2(f,x)\|w\|\leq \frac{1}{2}$, then for any $y\in \mathbb{C}^n$, either
			\[\theta\leq \varphi\leq \frac{\pi}{2}\text{ and }\left\|\mathcal{A}^{-1}f(y)\right\|\geq \frac{2}{\sqrt{3}}\gamma_2\|w\|\left(\frac{\sin\theta}{2\gamma_2}-\|w\|\right),\]
			or
			\[0\leq \varphi \leq \theta \text{ and }\left\|\mathcal{A}^{-1}f(y)\right\|\geq 4\gamma_2^2\|w\|\left(\frac{\sin\theta}{2\gamma_2}-\|w\|\right).\]
		\end{lem}
		\begin{proof}
			Let $\theta\leq \varphi\leq \frac{\pi}{2}$. Then, by lemma \ref{lem:bigAngleCase}, we have
			{\footnotesize
				\begin{eqnarray*}
					\sqrt{3}\left\|\mathcal{A}^{-1}f(y)\right\| & = &  \left\|\left(\begin{array}{ccc}
						\frac{1}{\sqrt{3}}D\hat{f}(x)^{-1} & 0 & 0\\
						0 & \frac{\sqrt{3}}{d_{n-1}^2(f_{n-1})} & 0 \\
						0 & 0 & \frac{\sqrt{3}}{d_{n}^2(f_{n})}
					\end{array}\right)\right\|\geq \left\|D\hat{f}(x)^{-1}\hat{f}(y)\right\|\\
					&\geq & \|w\|\sin\varphi -2\hat{\gamma}_2(f,x)\|w\|^2 \geq 2\gamma_2\|w\|\left(\frac{\sin\theta}{2\gamma_2}-\|w\|\right)
				\end{eqnarray*}}
				Now, let $0\leq \varphi \leq \theta$. By lemma \ref{lem:inverseOperatorLowerbound}, we have
				{\footnotesize
					\[\left\|\mathcal{A}^{-1}f(y)\right\|\geq \sum\limits_{j=n-1}^n\left(\left(\frac{1}{2}\cos^2\varphi_{j}-\gamma_{2}\cos\varphi_j\sin\varphi_j-\frac{1}{2}\gamma_{2}\sin^2\varphi_j\right)\|w\|^2 -2\gamma_{2}^2\|w\|^3\right).\]}
				Let 
				{\footnotesize
					\[h(\varphi) = \frac{\cos^2 \varphi - 2\gamma_2 \cos\varphi \sin\varphi - \gamma_2\sin^2\varphi}{4\gamma_2^2}.\]}
				We claim that 
				\[h(\theta)\geq \frac{\sin\theta}{\gamma_2}.\]
				Since $\sin\theta=\frac{d}{\gamma_2}$, it is enough to show that
				{\footnotesize
					\begin{equation}\label{eq:h_inequality}
					\left(1-\frac{d^2}{\gamma_2^2}\right)-2d\sqrt{1-\frac{d^2}{\gamma_2^2}}-\frac{d^2}{\gamma_2}-4d\geq 0.
					\end{equation}}
				For any fixed value of $d\in \left[0,\frac{1}{2}\right]$, the function of $\gamma_2$ in (\ref{eq:h_inequality}) is increasing when $\gamma_2\geq 1$\todo{check}. Therefore, it suffices to check this inequality is true for $\gamma_2=1$. Since, $d\approx 0.158921$ is the smallest positive root of
				\[(1-d^2)-2d\sqrt{1-d^2}-d^2-4d\]
				which lies in $\left[0,\frac{1}{2}\right]$, we have $h(\theta)\geq\frac{\sin\theta}{\gamma_2}$.
				
				Moreover, since $\gamma_2\geq 1$, we have $\sin \theta \leq d$, and so $\theta \leq \arcsin d$. Since $h(\varphi)$ is nonnegative and decreasing for $\varphi\in \left[0,\arcsin d\right]$, if $0\leq \varphi \leq \theta$, then $h(\varphi)\geq h(\theta)\geq\frac{\sin \theta}{\gamma_2}$. Therefore, by lemma \ref{lem:sinIneq} and \ref{lem:inverseOperatorLowerbound},
				{\footnotesize \begin{eqnarray*}
						\left\|\mathcal{A}^{-1}f(y)\right \|&\geq& \sum\limits_{j=n-1}^n 2\gamma_2^2\|w\|^2\left(h(\varphi_{j})-\|w\|\right)\geq 2\gamma_2^2\|w\|^2\left(\frac{\sin\varphi_{n-1}+\sin\varphi_{n}}{\gamma_2}-2\|w\|\right)\\
						& \geq & 4\gamma_2^2\|w\|^2\left(\frac{\sin\varphi}{2\gamma_2}-\|w\|\right)
					\end{eqnarray*}}
				\end{proof}
\begin{proof}[Proof of Theorem \ref{thm:tripleRootSeparationBound}]
	Let $w=y-x$ as above. Since $f(y)=0$, if $\gamma_2\|w\|\leq \frac{1}{2}$, then by lemma \ref{lem:operatorlowerbound}, we have
	\[\|y-x\|=\|w\|\geq\frac{\sin\theta}{2\gamma_2}=\frac{d}{2\gamma_2^2} .\]
	In the case of $\gamma_2\|w\|\geq \frac{1}{2}$, the claim holds because $d<1$ and $\gamma_2\geq 1$.
\end{proof}
\begin{proof}[Proof of Theorem \ref{thm:lowerboundfy}]
	Note that 
	\[\|w\|=\|y-x\|\leq \frac{d}{4\gamma_2^2}=\frac{\sin\theta}{4\gamma_2}.\]
	Therefore, by lemma \ref{lem:inverseOperatorLowerbound}, we have
	\[\left\|\mathcal{A}^{-1}f(y)\right\|\geq 4\gamma_2^2\|w\|^2\left(\frac{\sin\theta}{2\gamma_2}-\|w\|\right)\geq4\gamma_2^2\|w\|^2\frac{\sin\theta}{4\gamma_2}=d\|w\|^2\]
\end{proof}
\begin{proof}[Proof of Theorem \ref{thm:multiplicityOfDiffSystem}]
	Note that for any $y$ satisfying $\|y-x\|=R$, we have
	\[\|f(y)-g(y)\|\leq d_R(f,g)<\frac{dR^2}{\left\|\mathcal{A}^{-1}\right\|}\leq \|f(y)\|\]
	by theorem \ref{thm:lowerboundfy}. Then, by Rouch\'e's theorem, $f$ and $g$ have the same number of zeros inside $B(x,R)$. Since $x$ is the only root of $f$ inside $B(x,R)$ by theorem \ref{thm:tripleRootSeparationBound}, $g$ has three zeros in $B(x,R)$.
\end{proof}
\begin{proof}[Proof of Theorem \ref{thm:clusterthm}]
	Note that $g(x)=0$ from the definition of $g(X)$. Also, 
	\[Dg(x)=Df(x)-H=\begin{bmatrix}
	D\hat{f}(x) & 0 \\
	0 & 0
	\end{bmatrix}.\]
	Moreover, we have 
	\begin{eqnarray*}
	d_j(f_j) = \frac{\partial f_{j}}{\partial X_j} - \frac{\partial f_{j}}{\partial X_j} = 0,\\
	d_j^2(f_j) = \frac{\partial^2 f_{j}}{\partial X_j^2}\ne 0.
	\end{eqnarray*}
	It means that $Dg(x)$ satisfies the normalized form, and $x$ is a simple triple root with corank $2$ for $g$. Now, let $R=\frac{d}{4\gamma_2^2}$, then we find that 
	\begin{eqnarray*}
	d_R(g,f)&=&\max\limits_{\|y-x\|\leq R}\|g(y)-f(y)\|\\
	& \leq & \|f(x)\|+\|H\|R\\
	& = & \|f(x)\|+\|H\|\frac{d}{4\gamma_2^2}.
	\end{eqnarray*}
	Now, suppose that 
	\[\|f(x)\|+\|H\|\frac{d}{4\gamma_2^2}<\frac{d^3}{16\gamma_2^4\|\mathcal{A}^{-1}\|}.\]
	Then, $d_R(g,f)< \frac{dR^2}{\|\mathcal{A}^{-1}\|}$, and so by theorem \ref{thm:multiplicityOfDiffSystem}, $f$ has three zeros (counting multiplicities) in the ball $B(x,R)$.
\end{proof}
\bibliography{ref.bib}
\bibliographystyle{plain}
	


\end{document}
